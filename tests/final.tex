\documentclass[usletter]{article}
\usepackage{graphicx}
\usepackage{amsfonts}
\usepackage{amsthm}
\usepackage{amsmath}
\usepackage{amssymb}
\usepackage{test}
\usepackage[margin=1.5in]{geometry}

\begin{document}

\makeheader{John Bender}{December 16, 2014}{1}{Final Examination}

\begin{enumerate}
  \item Prove that $\mathsf{disc}(\mathsf{DISJ_n}) \leq O(\frac{1}{\sqrt{n}})$.
    \begin{proof}
      We know from the non-determinism barrier that:

      \begin{equation*}
        \log \frac{1 - 2 \cdot \frac{1}{3}}{\mathsf{disc}(\mathsf{DISJ_n})} \leq \log n + o(1)
      \end{equation*}

      Then since $\log n = \Theta(\log \sqrt{n})$:

      \begin{equation*}
        \log \frac{1 - 2 \cdot \frac{1}{3}}{\mathsf{disc}(\mathsf{DISJ_n})} \leq O(\log \sqrt{n})
      \end{equation*}

      And as a result $\mathsf{disc}(\mathsf{DISJ_n}) \leq O(\frac{1}{\sqrt{n}})$
    \end{proof}

  \item Prove that $R_{1/3}(\mathsf{IP}_n) \geq n - O(1)$.

    We know that the discrepancy method can only give us $\frac{n}{2}$. It seems like the pattern matrix method would give us $\Omega(n)$ which isn't an improvement so we expect that the corruption bound or the generalized discrepancy method are the only ways to go.

  \item Use the approach of convex relaxations to derive the generalized discrepancy method.

    Borrowing from the original discrepancy method proof:

    \begin{equation*}
      \exists \phi . \sum_{x,y} (1 - 2\epsilon) \cdot |\phi(x,y)| >
        2^c \max_{R} \left\{ \sum_{f^{-1}(1) \cap R} |\phi(x,y)| - \sum_{f^{-1}(0) \cap R} | \phi(x,y)| \right\}
    \end{equation*}

    On the left hand side with some manipulation and by linearity:

    \begin{align*}
      & \sum_{x,y} 1 |\phi(x,y)| - 2\epsilon \sum_{x,y} |\phi(x,y)| \\
      = & \sum_{x,y} 1 |\phi(x,y)| - 2\epsilon ||\phi||_1
    \end{align*}

    From there we need to alter the relaxation of $F$ so that $\sum_{x,y} 1 |\phi(x,y)|$ can be seen as $\sum_{x,y} -1^{f(x,y)} \cdot |\phi(x,y)|$ or $\langle -1^{f(x,y)}, \phi \rangle$. This could be achieved by changing the relaxation to be $M_(x,y) \geq -1-2\epsilon$ for $f^{-1}(0)$, but it's not clear if the sets are still disjoint under that condition.

  \item Prove that the choice of error parameter $0 < \epsilon < \frac{1}{2}$ affects the approximate degree of a Boolean function by at most a multiplicative constant \ldots

    \begin{enumerate}
      \item We know that $R_{\frac{1}{2} - \delta}(f) \cdot \lceil \frac{1}{2\delta^2} \cdot \ln \frac{1}{\epsilon} \rceil \geq R_{\epsilon}(f)$, by error reduction.
      \item We know that $R_{\epsilon}(f) \geq \Omega(\mathsf{deg}_{\frac{1}{3}}(f))$, from the pattern matrix method.
    \end{enumerate}

    \begin{proof}
      Assume that there exists a function $f$ such that no two constants exist that satisfy $c_{\epsilon,\delta}\mathsf{deg}_{\epsilon}(f) \leq \mathsf{deg}_{\delta}(f) \leq C_{\epsilon,\delta}\mathsf{deg}_{\epsilon}(f)$. Without loss of generality we assume that $\mathsf{deg}_{\epsilon}(f) = \Omega(\mathsf{deg}_{\delta}(f))$ which means that we should select $\delta$ for our error when computing $f$, because the polynomial approximation at that error will require fewer bits of communication by (b). But then this contradicts (a) which implies there is always a constant factor of difference in communication between two errors for the same function.
    \end{proof}


  \item Let M be a nonnegative matrix \ldots

    \begin{claim}
      $D(f) \leq \ln(rk_+\ M)$
    \end{claim}

    It's known that $rk\ M_f \leq rk_+\ M_f$ \cite{wikipedia} so we must show that $2^{D(f)}$ fits somewhere in that gap.

  \item A village has $n$ residents \ldots

    \begin{claim}
      After $k$ days, there will be $n - k$ residents. Since $k \leq n$, this implies that on the $(n + 1)$th day $n - k$ residents will remain.
    \end{claim}

    \begin{proof}
      We proceed by induction on the number of marked residents. Under the assumption that at least one resident is marked, then if only one resident is marked, she will see that no other residents have marks and conclude that she must leave at the end of the first day. By the inductive hypothesis, on the $k$th day $k - 1$ residents should have left if there are only $k - 1$ marks. Otherwise, if a resident sees $k - 1$ other marks on the $k$th day, she must conclude that she is marked and then will leave before the next day (note, that this applies to all of the $k$ marked residents).
    \end{proof}
\end{enumerate}

\newpage

\bibliography{1}
\bibliographystyle{plain}

\end{document}
