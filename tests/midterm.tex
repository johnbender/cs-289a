\documentclass[usletter]{article}
\usepackage{graphicx}
\usepackage{amsfonts}
\usepackage{amsthm}
\usepackage{amsmath}
\usepackage{amssymb}
\usepackage{test}
\usepackage[margin=1.5in]{geometry}

\begin{document}

\makeheader{John Bender}{November 10, 2014}{1}{Midterm Examination}

\begin{enumerate}
  \item Let $f : \{0,1\}^n \times \{0,1\}^n \rightarrow \{0,1\}$ be given by $f(x,y) = 1 \Leftrightarrow \Sigma x_i y_i \equiv 0\ (\mathsf{mod}\ 18181)$. Prove that $f$ has no fooling set larger than $n^c$, for some constant $c$.


    TODO double check assumption that this is inner product (because the prime is odd the 0 ~ 1 with $F_2$)
    TODO the constant?

    \begin{proof}
      We know the following:

      \begin{enumerate}
        \item $M_{\mathsf{IP}_n} = \left [ \sum_{i=1}^{n} x_i y_i \right ]$ in $\mathbb{F}_2$
        \item $rk_{\mathbb{F}_2}\ M_{\mathsf{IP}_n} = n$
        \item $\forall \mathbb{F}, \mathbb{K}, rk_{\mathbb{F}}\ M = rk_{\mathbb{K}}\ M$ where $\mathbb{F}, \mathbb{K}$ are finite fields.
        \item $\forall \mathbb{F}, f.|fs(f)| \leq (1 + rk_{\mathbb{F}}\ M_f)^2$ where $M_f$ is the characteristic matrix for $f$.
      \end{enumerate}

      From the definition and (a) $f$ has can be seen as $\left [ \sum_{i=1}^{n} x_i y_i \right ]$ in $\mathbb{F}_{18181}$. Then by (b) and (c) we know that $M_f$ has rank $n$ and taken with (d) this implies that the size of the fooling set is certainly no greater than $n^3$.
    \end{proof}


\end{enumerate}

\newpage

\bibliographystyle{abbrv}
\bibliography{1}

\end{document}
