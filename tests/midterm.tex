\documentclass[usletter]{article}
\usepackage{graphicx}
\usepackage{amsfonts}
\usepackage{amsthm}
\usepackage{amsmath}
\usepackage{amssymb}
\usepackage{test}
\usepackage[margin=1.5in]{geometry}

\begin{document}

\makeheader{John Bender}{November 10, 2014}{1}{Midterm Examination}

\begin{enumerate}
  \item How much communication do Alice and Bob need to find out deterministically if their graphs are isomorphic?

    \begin{claim}
      $O(n)$ bits are required.
    \end{claim}

    \begin{proof}
      We generously assume that graphs can be encoded in $|V| \cdot log(|V|)$ bits, that is without the edges at all. Then for input size $n$ we have the following:

      \begin{equation}
        |V| = \frac{n}{log(|V|)}
      \end{equation}

      We also know the following:

      \begin{enumerate}
        \item In the characteristic matrix $M_{\mathsf{ISO_n}}$ each row will have at most $\left (\frac{n}{log(|V|)}\right)!$ entries for all the re-orderings of the elements of $|V|$.
        \item For every row $G_i$ where the entry at $G_{i,j}$ is one, there is an identical row for $G_j$ because $G_j$ will be isomorphic to the same graphs as $G_i$.
      \end{enumerate}

      By (a) and (b) we know that that rank of $M_{\mathsf{ISO_n}}$ can be at most:

      \begin{equation}
        2^n - \left (\frac{n}{log(|V|)}\right)!
      \end{equation}

      Which is essentially $2^n$ for sufficiently large $n$. Also, if the number of ones in a given row is essentially $O(1)$ then we would need at least $2^n$ other rows to find another entry at any given position in the matrix. So for sufficiently large $n$ we have:

      \begin{equation}
        rk_{\mathbb{F}_2} M_{\mathsf{ISO_n}} = O(2^n)
      \end{equation}

      And by the rank bound we have that $D(\mathsf{ISO_n}) = n + 1$ which is tight.
    \end{proof}

  \item Construct a $\mathsf{P}^{\mathsf{cc}}$ communication problem.
    \begin{proof}
      We know the following:

      \begin{enumerate}
        \item A problem $\{f_n\} \in \mathsf{P}^{\mathsf{cc}}$ has a protocol tree of height at most $\log n$ by definition.
        \item The number of nodes in the protocol tree is at most $2^{\log n} - 1$ by (a).
        \item The number of edges in the protocol tree is at most $2^{\log n} - 2$ by (a).
      \end{enumerate}

      From the above we can define $A_n(x), B_n{x}$ to be an encoding of the protocol tree at $x$ and $y$ respectively, which requires at most $\log(2 \cdot 2^{\log n}) = \log n + 1$ by (b) and (c). Then the protocol for our complete problem is just a coordinated traversal of the tree where left is $0$ and right is $1$ requiring at most $\log n$ single bit communications.

      We know that this protocol is in $\mathsf{P}^{\mathsf{cc}}$ since it requires $\log n$ bits of communication and we also know that the range of the reductions is $\{0,1\}^{2^{\log n}}$ by the encoding.
    \end{proof}

  \item Let $f : \{0,1\}^n \times \{0,1\}^n \rightarrow \{0,1\}$ be given by $f(x,y) = 1 \Leftrightarrow \Sigma x_i y_i \equiv 0\ (\mathsf{mod}\ 18181)$. Prove that $f$ has no fooling set larger than $n^c$, for some constant $c$.


    TODO double check assumption that this is inner product (because the prime is odd the 0 ~ 1 with $F_2$) \\
    TODO the constant?

    \begin{proof}
      We know the following:

      \begin{enumerate}
        \item $M_{\mathsf{IP}_n} = \left [ \sum_{i=1}^{n} x_i y_i \right ]$ in $\mathbb{F}_2$
        \item $rk_{\mathbb{F}_2}\ M_{\mathsf{IP}_n} = n$
        \item $\forall \mathbb{F}, \mathbb{K}, rk_{\mathbb{F}}\ M = rk_{\mathbb{K}}\ M$ where $\mathbb{F}, \mathbb{K}$ are finite fields.
        \item $\forall \mathbb{F}, f.|fs(f)| \leq (1 + rk_{\mathbb{F}}\ M_f)^2$ where $M_f$ is the characteristic matrix for $f$.
      \end{enumerate}

      From the definition and (a) $f$ has can be seen as $\left [ \sum_{i=1}^{n} x_i y_i \right ]$ in $\mathbb{F}_{18181}$. Then by (b) and (c) we know that $M_f$ has rank $n$ and taken with (d) this implies that the size of the fooling set is certainly no greater than $n^3$.
    \end{proof}


  \item What is the nondeterministic communication complexity of $f$ in the previous problem?

    We know the following:

    \begin{enumerate}
      \item $D(\mathsf{IP}_n)$ is $n$ from notes
      \item $N(\mathsf{IP}_n) \leq n - 1$ from notes
      \item $N(\mathsf{\lnot IP}_n) \geq n$ from notes
      \item Uniform distribution proof book page 12
    \end{enumerate}

    Super polylogarithmic since IP is not in $\mathsf{NP}^{cc}$, must be  $N(f) \leq D(f)$ due to construction of any ``guessing'' non-deterministic protocol. Must be $O(n^3)$? Looking for exact complexity it seems?

  \item On input linear subspaces $A, B \subseteq \mathbb{F}_2^n$ prove that $\Theta(n^2)$ bits of nondeterministic communication are necessary and sufficient to check if $A$ and $B$ are orthogonal.

    \begin{proof}
      We know the following:

      \begin{enumerate}
        \item Both $A$ and $B$ can be represented as $m$ (which is order $n$) linearly independent vectors of order $n$ bits ($O(n^2)$).
        \item $A$ and $B$ are orthogonal if $\forall x \in A, y \in B. \langle x, y \rangle = 0$ \cite{orthsub}
        \item $N(\mathsf{IP_n}) = n + 1$
      \end{enumerate}

      By (a) we know that $O(n^2)$ bits is sufficient to check the orthogonality of the subspaces because we can send all the vectors in the linearly independent subset of $A$ and compare with the same subset of $B$.

      By (b) and (c) we know that $o(n^2)$ bits is necessary since

    \end{proof}

\end{enumerate}

\newpage

\bibliography{1}
\bibliographystyle{plain}

\end{document}
