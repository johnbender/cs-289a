\documentclass[usletter]{article}
\usepackage{graphicx}
\usepackage{amsfonts}
\usepackage{amsthm}
\usepackage{amsmath}
\usepackage{amssymb}
\usepackage{scribe}

\newcommand{\matrixB}[0]{
    \begin{bmatrix}
      0 & 0 \\
      1 & 1 \\
    \end{bmatrix}
}

\begin{document}


\makeheader{John Bender}{October 10, 2014}{3}{Fooling Set Efficacy and The Rank Bound}

\noindent

Fooling sets as defined and described in lecture 2 are one technique for establishing a lower bound on the complexity of the communication protocol for deterministic functions. Here we examine a few examples that highlight how well and how poorly the technique works and we'll also see that, in many cases, the fooling set technique will not provide a good lower bound. We will conclude by introducing the rank bound technique and more examples.

\section{Fooling Set Examples} \label{sec:fooling-set-eg}

\begin{example}
  Greater Than
  \begin{align*}
    & \mathtt{GT}_n : \{0,1\}^n \times \{0,1\}^n \rightarrow \{0,1\}^n \\
    & \mathtt{GT}_n(x,y) = \left \{
      \begin{array}{ll}
        1  & \mathsf{if} x \geq y \\
        0 & \mathsf{otherwise}
        \end{array}
      \right.
  \end{align*}
\end{example}

First we look at $\mathtt{GT}_n$ which is $1$ when $x \geq y$ for using the integer representations of $x$ and $y$. We can easily choose the fooling set to be $S = \{ (z, z)\ |\ z \in \{0,1\}^n \}$, the set of pairs of inputs with the same bit vector. To see why this is a fooling set consider the characteristic matrix of $\mathtt{GT}_n$ where the rows and columns are ordered by their integer value (note: the fooling set is bolded).

\begin{equation*}
  M_{\mathtt{GT}_n} =
  \begin{bmatrix}
  \textbf{1} & 0 & \cdots & 0 \\
  1 & \textbf{1} & \cdots & 0 \\
  \vdots & \vdots & \ddots & \vdots \\
  1 & 1 & \cdots & \textbf{1}
  \end{bmatrix}
\end{equation*}

\vspace{0.3cm}

This clearly forms a fooling set since each element maps to the same result, $f(z,z) = 1$, and given two pairs $(z_1, z_1), (z_2, z_2)$ either $(z_1, z_2)$ or $(z_2, z_1)$ is in the top right of the matrix and has the value zero. Further since the size of $S$ is $|X|$ and as we saw previously each element in the fooling set must exist in its own f-monochromatic rectangle then, $D(\mathtt{GT_n}) \geq \lceil{log(|S|)}\rceil = n$. We can do slightly better if we also count 0-rectangles $D(\mathtt{GT_n}) \geq \lceil{log(|S| + 1)}\rceil = n + 1$. We know this is tight since the simplest protocol for $\mathtt{GT_n}$ involves one of the participants passing the entirety of their input $|x| \in X = n$ to the other and then a one bit response.

\begin{example}
  Disjointness
  \begin{align*}
    & \mathtt{DISJ}_n : \mathcal{P}({1,\ldots,n}) \times \mathcal{P}({1,\ldots,n}) \rightarrow \{0,1\}^n \\
    & \mathtt{DISJ}_n(x,y) = \left \{
      \begin{array}{ll}
        1  & \mbox{if } x \cap y = \varnothing \\
        0 & \mbox{otherwise}
        \end{array}
      \right.
  \end{align*}
\end{example}

Next we consider $\mathtt{DISJ}_n$, that is the function producing $1$ when $x,y \in \mathcal{P}({1,\ldots,n})$ such that $x \cap y = \varnothing$. Again we select a straight forward fooling set $S = \{(X, \overline{X})\ |\ X \in \mathcal{P}({1,\ldots,n})\}$ where $\overline{X}$ is the compliment of the set $X$. To see that this is a fooling set consider that for any two members of the fooling set, say $(A, \overline{A})$ and $(B, \overline{B})$, $A$ and $B$ must be different. That is,

\begin{equation*}
  \exists z.(z \in A \land z \notin B) \lor (z \in B \land z \notin A)
\end{equation*}

Either case implies that $z$ is in the compliment of $B$ or $A$ respectively, which in turn implies that a pair at the corner of the corresponding rectangle $(A, \overline{B})$ (resp. $(B, \overline{A})$) is not disjoint and the value of the disjointness function would be 0 as required. We conclude that the complexity is bounded from below by $\lceil{log(2^n + 1)}\rceil = n + 1$ which is again tight due to the trivial implementation of the protocol.

\begin{example}
  Inner Product
  \begin{align*}
    & \mathtt{IP}_n : \{0,1\}^n \times \{0,1\}^n \rightarrow \{0,1\} \\
    & \mathtt{IP}_n(x, y) = \bigoplus_{i=1}^n x_i \land y_i
  \end{align*}
\end{example}

Finally we examine the inner product function, $\mathtt{IP}_n$, where the fooling set method fails to provide a good lower bound. $\mathtt{IP}_n$ takes the bitwise $\land$ of each bit position from two bit vectors of length $n$ and then performs a fold over the resulting bit vector with $\oplus$. The output of the function is the single bit result.

To construct the fooling set for $\mathtt{IP}_n$ we pick one of the $n$ bits, flip it and pair it with its duplicate, i.e. $S = \{ (x', x') | i \in 1 \ldots n, x' = \langle 0_1, \ldots 0_{i-1},1,0_{i+1}, \ldots, 0_n \rangle \}$. Each of these pairs clearly results in the desired value and when paired with an other $x'$ the $\land$ of each bit will result in all zeros due to the unique position of the flipped bit in each vector. Unfortunately, because we can only reliably use a single bit to achiever our ends there are only $n$ pairs which gives us the weaker lower bound $log_2(n+1)$.

\section{Fooling Sets For Random Functions} \label{sec:fooling-set-random}

Since we have a sense that fooling sets might not work particularly well for some functions it might be nice to get a better picture of how frequently they are effective for finding a lower bound. We will consider the probability that a fooling set for a random function meets some basic threshold. In particular we will prove the following theorem.

\begin{theorem}
For a random function $f : \{0,1\}^n \times \{0,1\}^n \rightarrow \{0,1\}$ and the function that takes functions to their fooling sets, $fs : (\{0,1\}^n \times \{0,1\}^n \rightarrow \{0,1\}) \rightarrow S$, we have

\begin{equation*}
\Pr\left[ |fs(s)| \geq 10n \right] \leq \left(\frac{2}{3}\right)^{n^{2}}
\end{equation*}
\end{theorem}

That is, we will show that the probability that the size of the fooling set exceeds $10n$ for most random functions is very low. This is also the probability that a corresponding protocol will have a lower bound of greater than $log_2(10n)$. Coupled with the high probability we derived in the first lecture that the complexity of a given protocol will be greater than or equal to $n$ (the size of the inputs) we will be left with low confidence in the fooling set approach.

\begin{proof}
  First we establish the probability that given $S$ is a fooling set for some function $f$. We start by fixing $S = \{(x_1, y_1), \ldots, (x_s, y_s)\}$ such that all $x$ are distinct and all $y$ are distinct and establishing a probability for whether that $S$ is a fooling set for $f$.

\begin{align*}
\Pr_f\left[  S \in \mathsf{FS}(f) \right] &=
  \left( \frac{2}{2^{|S|}} \right) \cdot \left( \frac{3}{4} \right)^{{|S| \choose 2}}
\end{align*}

Where $\mathsf{FS}(f)$ is the set of all fooling sets for $f$, the probability that $S$ is in that set is determined by the probability of two independent events that align with the definition of a fooling set. First, $\left( \frac{2}{2^{|S|}} \right)$ is the probability that the fooling sets combination of $x$s and $y$s all result in the same output result from $f$, i.e. all zeros or all ones. Second, $\left( \frac{3}{4} \right)^{{|S| \choose 2}}$ is the probability that the other corners of the rectangle formed by any two pairs of inputs from $S$ will contain the ``offending'' value. That is, there are three ways that the other two corners could be configured with at least one offending value (e.g. assuming the value for the pairs in $S$ is 1 then $(0,1)$, $(0,0)$, $(1,0)$) which we raise to the power of the number of ways to choose any two elements in $S$ to form the rectangle because for every pairing of elements in $S$ they must exhibit this property.

\begin{align*}
\Pr_f\left[  S \in \mathsf{FS}(f) \right]
  &= \left( \frac{2}{2^{|S|}} \right) \cdot \left( \frac{3}{4} \right)^{{|S| \choose 2}} \\
  &= 2 \left( \frac{1}{\sqrt{3}} \right)^{|S|} \cdot \left( \frac{3}{2} \right)^{|S|^2} \\
  &\ll \left( \frac{3}{2} \right)^{|S|^2}
\end{align*}

We also know that the number of candidate fooling sets for a function $f$ is ${2^n \choose |S|}^2$ where the two choices are for a set of $x$s and a set of $y$s. As a result we can establish the probability that $f$ has a fooling set $S$ of at least $|S|$.

\begin{align*}
\Pr_f\left[ \exists S. S \in \mathsf{FS}(f) \land |S| > x \right]
  &= \Pr_f\left[ \exists S. S \in \mathsf{FS}(f) \land |S| = x \right] \\
  &= {2^n \choose x}^2 \cdot \left( \frac{3}{2} \right)^{x^2} \\
  &= 2^{2nx - log_2(\frac{2}{\sqrt{3}})x^2} \\
  &< \left( \frac{2}{3} \right)^{n^2} \mathsf{for}\ x = 10n
\end{align*}
\end{proof}

So we are left with very little in the way of confidence with the fooling set if we wish to prove a lower bound for a function we are interested in. Next we'll examine a different technique that relies on establishing the rank of the characteristic matrix for the function under consideration.

\section{Matrix Rank} \label{sec:matrix-rank}

\begin{definition}
  Matrix rank is the size of the largest set of linearly independent rows (or columns) for a given matrix. For $M \in F^{n \times m}$ where $F$ is an arbitrary field $rk_{F}(M) = min \{ k' : M = AB, A \in F^{n \times k}, B \in F^{k \times m} \}$. When the field subscript is omitted the field is assumed to be the reals.
\end{definition}

We are primarily interested in boolean matrices (i.e. matrices over the field $F_2$) and the impact of selecting this (or other) fields will be considered in later lectures. To make this distinction clear consider the following matrix:

\begin{example}
Matrix Rank:
\begin{equation*}
  M =
  \begin{bmatrix}
    1 & 1 & 1 & 1 \\
    1 & 1 & 0 & 0 \\
    1 & 0 & 1 & 0 \\
    1 & 0 & 0 & 1
  \end{bmatrix}
\end{equation*}
\end{example}

Here the rank of the matrix over $F_2$ is 3 since the first row can be constructed as a linear combination of the last three (recall that $1 + 1 + 1 = 0$ in $F_2$), but in the rank of the matrix over the reals is four since none of the rows can be constructed using a combination of the others.

Further consider the identity matrix and the matrix of ones, where the rank values are over any field.

\begin{example}
Identity and Ones Matrices:
\begin{align*}
  I_n =
  \begin{bmatrix}
  1 & 0 & \cdots & 0 \\
  0 & 1 & \cdots & \vdots \\
  \vdots & \vdots & \ddots & 0 \\
  0 & \cdots & 0 & 1
  \end{bmatrix}
  &&
  J_n =
  \begin{bmatrix}
  1 & 1 & \cdots & 1 \\
  1 & 1 & \cdots & \vdots \\
  \vdots & \vdots & \ddots & 1 \\
  1 & \cdots & 1 & 1
  \end{bmatrix}
  \\
  \\
  rk_{F}(I_n) = n
  &&
  rk_{F}(J_n) = 1
\end{align*}
\end{example}

Next we define two operations of interest in later discussions and how the rank is affected by those operations.

\begin{definition}
  Tensor Product:
  \begin{equation*}
    A \otimes B = [A_{i,j}B_{i',j'}]_{(i, i'),(j,j')}
  \end{equation*}
\end{definition}

\begin{example}
  Tensor Product Example:
  \begin{equation*}
    \begin{bmatrix}
      0 & 1 \\
      1 & 0 \\
    \end{bmatrix}
    \otimes
    \matrixB
    =
    \begin{bmatrix}
      0 \cdot \matrixB & 1 \cdot \matrixB \\
      1 \cdot \matrixB & 0 \cdot \matrixB
    \end{bmatrix}
  \end{equation*}
\end{example}

\begin{definition}
  Entry-wise Product:

  \begin{equation*}
    A \cdot B = [A_{i,j}B_{i,j}]
  \end{equation*}
\end{definition}

\begin{example}
  Entry-wise Product Example:
  \begin{equation*}
    \begin{bmatrix}
      0 & 1 \\
      1 & 0 \\
    \end{bmatrix}
    \cdot
    \matrixB
    =
    \begin{bmatrix}
      0 \cdot 0 & 1 \cdot 0 \\
      1 \cdot 1 & 0 \cdot 1
    \end{bmatrix}
  \end{equation*}
\end{example}

In addition to these operations we have the following properties:

\begin{fact}
\begin{alignat*}{3}
  &rk_{F}(c \cdot A)   && = \, \, &&rk_{F}(A)\ \mathsf{where}\ c \in F, c \neq 0 \\
  &rk_{F}(A + B)       &&\leq &&rk_{F}(A) + rk_{F}(B) \\
  &rk_{F}(A + B)       &&\geq &&|rk_{F}(A) - rk_{F}(B)| \\
  &rk_{F}(A \otimes B) &&= &&(rk_{F}(A))(rk_{F}(B)) \\
  & A^{\otimes n}      &&= &&A \otimes \ldots \otimes A\ n\ \mathsf{times}
\end{alignat*}
\end{fact}

\begin{proposition}
  Rank of the Entry-wise Product
  \begin{equation*}
    rk_{F}(A \cdot B) \leq (rk_{F}(A))(rk_{F}(B))
  \end{equation*}
\end{proposition}

\begin{proof}
  The proof is clear once it is observed that the entry-wise produce of $A$ and $B$ is a sub-matrix of the tensor product of the same two matrices.
\end{proof}

\section{The Rank Bound} \label{sec:rank-bound}

As we saw the fooling set method for finding lower bounds on communication protocols isn't effective in some important situations. With that in mind we examine The Rank Bound method which uses the rank of the characteristic matrix of for a given function $f$ and its rectangles to establish a lower bound on the number of those rectangles.

\begin{definition}
  The characteristic matrix of $R = A \times B \subseteq X \times Y$ is constructed by including all rows and columns from the characteristic function of $f$ but only the values from the rectangle $R$.
  \begin{equation*}
    M_R =
    \begin{bmatrix}
      1 & \cdots & 1 & 0 & \cdots & 0 \\
      \vdots & \ddots & \vdots & \vdots & \ddots & 0 \\
      1 & \cdots & 1 & 0 & \cdots & 0 \\
      0 & \cdots & 0 & 0 & \cdots & 0 \\
      \vdots & \ddots & \vdots & \vdots & \ddots & 0 \\
      0 & \cdots & 0 & 0 & \cdots & 0
    \end{bmatrix}
  \end{equation*}

Here the rows and columns in $A$ and $B$ have been arranged into the top left corner (since they form a rectangle).
\end{definition}

\begin{fact}
 Due to the construction of the matrix,
  \begin{equation*}
    rk(M_R) = 1
  \end{equation*}
\end{fact}

\begin{theorem}
  For all fields $F$ and functions $f : \{0,1\}^n \times \{0,1\}^n \rightarrow \{0,1\}^n $
  \begin{equation*}
    D(f) \geq \lceil log_2(rk_{F}(M_f)) \rceil
  \end{equation*}
\end{theorem}

\begin{proof}
Let $c = D(f)$, then there are $2^c$ $f$-monochromatic rectangles $R_1 \ldots R_{2^c}$ such that,
\begin{equation*}
  X \times Y = \bigsqcup_{i=1}^{2^c} R_i
\end{equation*}

Then since they are disjoint this implies that the sum of the characteristic matrices for the $f$-monochromatic rectangles for $f$ is the characteristic matrix for $f$ itself. Intuitively since each is rectangle is pairwise disjoint summing any two of them can only ``fill-in'' the ``empty'' areas of the other matrix.

\begin{equation*}
  M_f = \sum_{i:f(R_i) = 1}^{2^c} M_{R_i} \quad
    \Rightarrow \quad rk_{F}(M_f) \leq \sum_{i:f(R_i) = 1}^{2^c} rk_{F}(M_{R_i})
\end{equation*}

By the property of matrix rank under addition we have the above implication. From this and the earlier observation that the characteristic matrix for each $R$ is rank one we can see that $rk_{F}(M_f) \leq |\{\, i\, |\, f(R_i) = 1 \}| \leq 2^c$ and we conclude that $D(f) = c >= log_2(rk_{F}(M_f))$.
\end{proof}

\begin{corollary}
  \begin{equation*}
    D(f) \geq \lceil log_2(rk_{F}(M_f) + 1) \rceil
  \end{equation*}
\end{corollary}


Which can be seen by considering, again, the zero rectangles.

\begin{equation*}
  2^c \geq |\{\, i\, |\, f(R_i) = 1 \}| + |\{\, i\, |\, f(R_i) = 0 \}|
\end{equation*}

\section{Rank Bound Examples} \label{sec:rank-bound-eg}

To get an intuition for how, and how well, this method works we examine three examples $\mathtt{EQ}_n$, $\mathtt{GT}_n$, $\mathtt{DISJ}_n$.

\begin{example} Equality.
  For the equality function we have the characteristic matrix such that,

  \begin{equation*}
    M_{\mathtt{EQ}_n} = I_{2^n}
  \end{equation*}

  We observed earlier that the rank of the identity matrix is exactly its width (or height) so $rk_{F}(M_{\mathtt{EQ}_n}) = 2^n$ which implies that $D(\mathtt{EQ}_n) \geq \lceil log(2^n + 1) \rceil = n + 1$. Finally we know that his is tight due to the trivial implementation of the protocol.
\end{example}

\begin{example} Greater Than.
  For the greater than function we have the characteristic matrix,

\begin{equation*}
  M_{\mathtt{GT}_n} =
  \begin{bmatrix}
  1 & 0 & \cdots & 0 \\
  1 & 1 & \cdots & 0 \\
  \vdots & \vdots & \ddots & \vdots \\
  1 & 1 & \cdots & 1
  \end{bmatrix}
\end{equation*}

Since the rank of this matrix is also $2^n$ we have the same inequality as in the previous example which is again tight.
\end{example}

\begin{example} Disjointness.
  We conclude with a slightly more complex example in $\mathtt{DISJ}_n$. Recall that

  \begin{equation*}
    \mathtt{DISJ}_{n}(x,y) = \bigwedge_{i=1}^{n} \overline{(x_i \land y_i)}
  \end{equation*}

  Intuitively this is the position wise $\land$ and negation of each bit position in the bit vectors $x$ and $y$ and then the fold of $land$ over the resulting bit vector. This produces the matrix,

  \begin{equation*}
    M_{\mathtt{DISJ}_n} = \left[ \bigwedge_{i=1}^{n} \overline{(x_i \land y_i)} \right]_{x,y \in {0,1}^n}
  \end{equation*}

  Which is the value of the previously defined bit operation at each position in the matrix for each pair of inputs. If we consider this from an algebraic point of view where $\land$ maps to multiplication and negation maps to subtraction from 1,  the characteristic matrix can be defined as,

  \begin{equation*}
    M_{\mathtt{DISJ}_n} = \left[ \prod_{i=1}^{n} (1 - x_i y_i) \right]_{x,y \in {0,1}^n}
  \end{equation*}

  We can show that this is the tensor power of the following matrix, by an inductive argument.

  \begin{equation*}
    \begin{bmatrix}
      1 & 1 \\
      1 & 0
    \end{bmatrix}^{\otimes n}
  \end{equation*}

  \begin{proof} We proceed by induction on the size of the matrix $M_{\mathtt{DISJ}_1}$. Base case,
    \begin{equation*}
      M_{\mathtt{DISJ}_1} =
        \begin{bmatrix}
          1 & 1 \\
          1 & 0
        \end{bmatrix}
      \end{equation*}

      And the inductive case,

      \begin{equation*}
        M_{\mathtt{DISJ}_1} =
        \begin{bmatrix}
          M_{\mathtt{DISJ}_{n-1}} & M_{\mathtt{DISJ}_{n-1}} \\
          M_{\mathtt{DISJ}_{n-1}} & 0
        \end{bmatrix}
      \end{equation*}

      You can see that this construction is correct by considering the places where the ones and zeros arise at points in the matrix. That is, in the bottom right corner of a given quadrant of the matrix there will at least one overlap of bits such that the negation of the $\land$ of each bit will produce a 0 and the $\land$ of all of the bits together will produce a 0. This can be reduced to the following,

      \begin{align*}
        M_{\mathtt{DISJ}_n} =
        \begin{bmatrix}
          1 & 0 \\
          1 & 0
        \end{bmatrix}
        \otimes
        M_{\mathtt{DISJ}_{n-1}}
        \Rightarrow
        M_{\mathtt{DISJ}_n} =
        \begin{bmatrix}
          1 & 0 \\
          1 & 0
        \end{bmatrix} ^{\otimes n}
      \end{align*}
  \end{proof}

      Finally from this we can say that $rk(M_{\mathtt{DISJ}_n}) = 2^n$ since the rank of the base case is 2 and we take the tensor produce n times concluding our examples.
\end{example}
\bibliographystyle{abbrv}
\bibliography{1}

\end{document}
